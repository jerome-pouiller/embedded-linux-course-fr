% This document is available under the Creative Commons Attribution-ShareAlike
% License; additional terms may apply. See
%   * http://creativecommons.org/licenses/by-sa/3.0/
%   * http://creativecommons.org/licenses/by-sa/3.0/legalcode
%
% Copyright 2010 Jérôme Pouiller <jezz@sysmic.org>
%

\part{Debuguer}

\begin{frame}
  \partpage
\end{frame}

\begin{frame}
  \tableofcontents[currentpart]
\end{frame}

\section{Debug}

\begin{frame}[fragile=singleslide]{Debuguer une application utilisateur}
  Utilisation classique.  \verb+-g3+ permet d'inclure  les symboles de
  debug
  \begin{lstlisting}
host$ gcc -Wall -g3 -c hello.c -o hello.o
host$ gcc -Wall -g3 hello.o -o hello-x86-dbg
host$ gdb hello-x86-dbg
  \end{lstlisting}  
\end{frame}

\begin{frame}[fragile=singleslide]{Debuguer une application utilisateur}{Commandes utiles}
  Gestion de l'éxecution
  \begin{itemize}
  \item Point d'arrêt à l'adresse \verb+0x9876+
    \begin{lstlisting}
gdb> b *0x9876
    \end{lstlisting}
  \item \verb+b  hello.c:30+ Point  d'arrêt à la  ligne 30  du fichier
    \file{hello.c}
    \begin{lstlisting}
gdb> b hello.c:30
    \end{lstlisting}
  \item Point d'arrêt sur la fonction \cmd{main}
    \begin{lstlisting}
gdb> b main
    \end{lstlisting}
  \item Continuer un programme arrêté
    \begin{lstlisting}
gdb> c
    \end{lstlisting}
  \item Démarrer le programme avec \verb+arg+ comme argument
    \begin{lstlisting}
gdb> r arg
    \end{lstlisting}
  \end{itemize}
\end{frame}
  
\begin{frame}[fragile=singleslide]{Debuguer une application utilisateur}{Commandes utiles}
  Obtenir des informations
  \begin{itemize}
  \item Voir la pile d'appel
    \begin{lstlisting}
gdb> bt
    \end{lstlisting}
  \item Afficher les variable locales à la fonction
    \begin{lstlisting}
gdb> i locales
    \end{lstlisting}
  \item Afficher les arguments de la fonction
    \begin{lstlisting}
gdb> i args
    \end{lstlisting}
  \item Afficher la valeur de \verb+a+
    \begin{lstlisting}
gdb> p a
    \end{lstlisting}
  \item Afficher la valeur de \verb+abs(2 * (a - 4))+
    \begin{lstlisting}
gdb> p abs(2 * (a -  4))
    \end{lstlisting}
  \end{itemize}
\end{frame}

\begin{frame}[fragile=singleslide]{Debuguer une application utilisateur}{Commandes utiles}
  Debuguer à chaud
  \begin{itemize}
  \item Attacher \verb+gdb+ à un processus existant
    \begin{lstlisting}
gdb> attach 1234
    \end{lstlisting}
  \item Arrêter le debug sans arrêter le processus
    \begin{lstlisting}
gdb> detach
     \end{lstlisting}
   \end{itemize}
\end{frame}

\begin{frame}[fragile=singleslide]{Debuguer une application utilisateur}{Commandes utiles}
  Et plus...
  \begin{itemize}
  \item Afficher  la valeur de \verb+i+  à chaque fois  que l'on passe
    sur le point d'arrêt 1
    \begin{lstlisting}
gdb> command 1
> silent
> p i
> c
> end
    \end{lstlisting}
  \item Arrêter le programme lorsque la variable \verb+i+ est modifiée
    ou lue
    \begin{lstlisting}
gdb> watch i
gdb> awatch i
      \end{lstlisting}
    \item Passer d'une thread à une autre
      \begin{lstlisting}
gdb> thread 2
gdb> thread 1
      \end{lstlisting}
    \end{itemize}
\end{frame}

\begin{frame}[fragile=singleslide]{Debuguer une application utilisateur}
  Utilisation des Cores Dump:
  \begin{itemize}
  \item Vérifiez la limite \emph{coredump size} avec \texttt{ulimit -c}
  \item Core Dump automatique en cas de crash
  \item Forcé avec \texttt{kill -QUIT} ou \verb|<Ctrl+\>| (man kill)
  \item Créé avec la commande gdb \texttt{gcore} (voir avec attach)
  \item  Il est possible  de savoir  à quelle  binaire est  attaché le
    coredump avec \cmd{file}
  \end{itemize}
\end{frame}

\begin{frame}[fragile=singleslide]{Debuguer une application utilisateur distante}
  Utilisation croisée
  \begin{itemize}
  \item Compilation
    \begin{lstlisting}
host$ arm-linux-gcc -Wall -g3 -c hello.c -o hello.o
host$ arm-linux-gcc -Wall -g3 hello.o -o hello-arm-dbg
    \end{lstlisting}
  \item Démarrage du serveur de debug
    \begin{lstlisting}
target$ gdbserver *:1234 hello
host$ arm-ulibceabi-gdb
    \end{lstlisting}
  \item Connexion au serveur de debug
    \begin{lstlisting}
> target remote target:1234
    \end{lstlisting}
  \item Configuration des chemins
    \begin{lstlisting}
> set solib-absolute-prefix /home/user/nfs-root
    \end{lstlisting}
  \item Chargement des symboles de debug
    \begin{lstlisting}
> exec-file hello
    \end{lstlisting}  
  \end{itemize}
\end{frame}

\begin{frame}[fragile=singleslide]{Debuguer une application utilisateur distante}
  Debug d'une application en cours d'éxecution
  \begin{itemize}
  \item Execution
    \begin{lstlisting}
target$ ./hello > /dev/null & 
    \end{lstlisting}  
  \item On attache \cmd{gdb} au processus
    \begin{lstlisting}
target$ gdbserver *:1234 --attach $!
    \end{lstlisting}  
  \item configuration classque de \cmd{gdb}
    \begin{lstlisting}
host$ arm-ulibceabi-gdb
> target remote target:1234
> set solib-absolute-prefix /home/user/nfs-root
> exec-file hello
    \end{lstlisting}  
  \item Il est peut-être intéressant de faire un \emph{core dump}
    \begin{lstlisting}
> gcore
    \end{lstlisting}  
  \item Libérons le processus
    \begin{lstlisting}
> detach
    \end{lstlisting}  
  \end{itemize}
\end{frame}

\begin{frame}[fragile=singleslide]{Debuguer une application utilisateur}
  Séparer les symbols  de debug permet:
  \begin{itemize}
  \item de n'uploader sur la cible que le nécéssaire à l'éxecution.
  \item  de  livrer  des  binaires  strippées en  production  tout  en
    conservant les possibilités de debug
  \item d'analyser des \emph{core dump} produit en production
  \end{itemize}
  Utilisation de \cmd{objcopy} pour créer le fichier de symboles:
  \begin{lstlisting}
host$ arm-linux-objcopy --only-keep-debug hello hello.sym
host$ chmod -x hello.sym
host$ arm-linux-strip --strip-unneeded hello 
target$ gdbserver *:1234 hello
host$ arm-linux-gdb
> target remote target:1234
> symbol-file hello.sym
    \end{lstlisting} 
\end{frame}
  
\begin{frame}[fragile=singleslide]{Debuguer une application utilisateur}
  Sous certaines conditions, il est possible de rendre le chargement
  des symboles automatique:
  \begin{lstlisting}
host$ arm-linux-objcopy --add-gnu-debuglink=hello.sym hello
  \end{lstlisting}  
  Remarque: l'option -g n'influence pas le code compilé:
    \begin{lstlisting}
$ gcc hello.c -O3 -o hello
$ gcc hello.c -O3 -g -o hello.dbg
$ strip hello.dbg hello
$ cmp hello.dbg hello && echo ok
ok
    \end{lstlisting}  % $
\end{frame}

\begin{frame}[fragile=singleslide]{Analyse sans execution}
  Très bons outils pédagogiques
  \begin{itemize}
  \item  \cmd{strings} affiche  les chaine  de  caractères affichables
    d'une binaire
  \item \cmd{ldd} affiche quel  sera le comportement de \verb+ld+ lors
    du chargement (en d'autres  termes, liste les dépendances avec des
    bibliothèques dynamique)
  \item \cmd{gdb} est  capable de charger des fichiers  objets et d'en
    sortir beaucoup d'informations sans execution
  \item  \cmd{objdump  -h} permet  d'obtenir  des  information sur  le
    placement des sections dans le fichier
  \item \cmd{readelf -s} ou \cmd{nm} liste les symbols contenu dans un
    fichier
  \item \cmd{objdump -d} permet de désassembler un fichier objet
  \item \cmd{objdump -S}  permet d'obtenir l'assembleur intermixé avec
    les code
  \item \cmd{objdump -C} permet de "demangler" des fonction C++
  \end{itemize}
% \note[item] {
% Ainsi:
%  Client:
%    Error at 0x8530
%  Séance de debug:
%     \begin{lstlisting}
%   host$ gdb hello
%   gdb> info file
%   # On remarque que 0x8530 se toruve dans la section .text
%   gdb> l *0x8530
%   # Ou bien directement
%   gdb> info line *0x8530
%     \end{lstlisting} 
% }
\end{frame}

% \begin{frame}[fragile=singleslide]{Utilisation de eclipse}
%     \begin{lstlisting}
% wget 'http://rm.mirror.garr.it/mirrors/eclipse/technology/epp/downloads/release/helios/SR1/eclipse-linuxtools-helios-SR1-incubation-linux-gtk-x86_64.tar.gz'
% ./eclipse 
% \end{lstlisting}
% \end{frame}

\begin{frame}[fragile=singleslide]{Debuguer le noyau}
  Activiation d \cmd{kgdb} permettant d'embarquer \cmd{gdbserver} dans
  le noyau:
  \begin{itemize}
  \item Compilation
    \begin{lstlisting}
host$ cp -a build build-dbg
host$ make O=build-dbg ARCH=arm CROSS_COMPILE=arm-linux- menuconfig
... Compile the kernel with debug info ...
... KGDB: kernel debugging with remote gdb ...
host$ make -j3 O=build-dbg ARCH=arm CROSS_COMPILE=arm-linux- uImage
    \end{lstlisting} 
  \item L'image ELF est beaucoup plus grosse
    \begin{lstlisting}
host$ ls -l build-dbg/vmlinux build/vmlinux
host$ cp build-dbg/uImage /srv/tftp/uImage-2.6.33.7-dbg
    \end{lstlisting} 
  \end{itemize}
\end{frame}
\begin{frame}[fragile=singleslide]{Debuguer le noyau}
  \begin{itemize}
  \item  Passage des arguments  \cmd{kgdboc} indiquant  quel interface
    \cmd{kgdb} doit utilisé et  \cmd{kgdbwait} indiquant que kgdb doit
    attendre notre connexion avant de continuer le boot
    \begin{lstlisting}
uboot> set bootargs ${bootargs} kgdboc=ttyS0 kgdbwait
uboot> tftp
<Ctrl-a q>
host$ arm-linux-gdb vmlinux 
gdb> target remote /dev/ttyUSB1
    \end{lstlisting} 
  \end{itemize}
\end{frame}

\begin{frame}[fragile=singleslide]{Debuguer le noyau}
  Nécessite  une modification  du driver  \file{atmel\_serial}  afin de
  permettre   l'utilisation  de   son  port   serie   par  \cmd{kgdb}.
  Modification  apportée par  Albin Tonerre  backportée de  la version
  2.6.34 du noyau (nous y reviendrons):
  \begin{lstlisting}
host$ patch -p1 < 0001-serial-atmel_serial-add-poll_get_char-and-poll_put_c.patch 
  \end{lstlisting} 

\note[item]{testé avec la version 2.6.37, mais devait fonctionner comme ca}
\end{frame}

% \begin{frame}[fragile=singleslide]{Debuguer avec Qemu}
%   \note[item]{il faudrait ajouter le support de l'AT91 dans qemu et ca devrait tourner plus ou moins}
%     \begin{lstlisting}
% host$ qemu -S -s -M integrator -cpu arm926 -m 16 -kernel zImage.integrator -initrd arm_root.img &
% host$ arm-linux-gdb vmlinux
% gdb> interrupt
%     \end{lstlisting} 
% \note[item]{TODO P3}
% \end{frame}

\begin{frame}[fragile=singleslide]{\cmd{ld.so}}
  Lors du chargement d'une binaire, la bibliothèque \cmd{ld.so} charge
  les bibliothèques indiquées dans  la section \emph{.dynamic}. Il est
  possible de  changer le comportement de \cmd{ld.so}  par des fichier
  de  configuration  (\file{/etc/ld.so.conf})  ou  par  des  variables
  d'environnement.
  \begin{itemize} 
  \item  \cmd{LD\_LIBRARY\_PATH} permet d'ajouter  un répertoire  ou les
    bibliothèques seront recherchées.
  \item \cmd{LD\_PRELOAD} permet de précharger une biliothèque.  Ainsi
    les  symboles  définit   dans  cette  bibliothèque  surcharge  les
    symboles standards:
    \begin{lstlisting}[language=c]
#include <stdio.h>
#include <unistd.h>
unsigned int sleep(unsigned int s) {
    printf("Try to reveal race-conditions\n");
    sleep(5 * s);
}
    \end{lstlisting} 
  \end{itemize}
\end{frame}  

\begin{frame}[fragile=singleslide]{\cmd{ld.so}}
  \begin{itemize} 
    \begin{lstlisting} 
host$ gcc -shared -fPIC mysleep.c -o libmysleep.so
target$ LD_PRELOAD=libmysleep.so ./hello 
target$ export LD_PRELOAD=libmysleep.so
taregt$ ./hello 
    \end{lstlisting} 
  \item Cette  technique est utilisé pour tracer  et profiler certains
    évènement
  \item Les expert en sécurité l'utilisent pour détourner un programme
    de son but original
  \item Enfin, il permet de  simuler des problèmes afin de vérifier la
    robustesse d'un programme
  \end{itemize} 
  \note[item]{TODO P4: Tester ERESI}
\end{frame}

\begin{frame}[fragile=singleslide]{strace et ltrace}
  \begin{itemize}
  \item \cmd{strace} permet de d'afficher les appels systèmes effectué
    par   un   processus.   Il   a  l'avantage   d'afficher   beaucoup
    d'informations sous des formes lisible.
    \begin{lstlisting} 
target$ strace ./hello
    \end{lstlisting}  
  \item  \cmd{ltrace}  est  une  génératlisation de  \cmd{strace}.  Il
    permet de suivre les appels à une bibliothèque extérieure
    \begin{lstlisting} 
host$ arm-linux-ldd --root . ./hello
target$ strace -l /lib/libc.so.0 -e sleep ./hello
target$ strace -l /lib/libc.so.0 -e malloc -e free /bin/ls
    \end{lstlisting}  
  \item Il est possible de filtrer les appels systèmes à suivre.
  \item  Ces outils  sont une  aide préciseuse  lorsque  l'on souhaite
    identifier le  comportement d'une  binaire dont nous  ne disposons
    pas des sources.
  \end{itemize} 
\end{frame}

\begin{frame}[fragile=singleslide]{ptrace}
  \cmd{ptrace}  est l'appel  système permettant  le  fonctionnement de
  debuguers et d'outils tel que \cmd{strace} et \cmd{ltrace}.
  \\[2ex]
  Il permet  de prendre la main  sur un processus,  puis d'obtenir des
  informations sur son état et de modifier sa mémoire.
\end{frame}

\section{Couverture}

\begin{frame}[fragile=singleslide]{Couverture de code}
  Principe:
  \begin{itemize}
  \item On instrumente le code généré
  \item Entre chaque ligne de C (équialent à plusieurs ligne
    d'assembleur), on ajoute une instruction du type \verb/inc $ADDR/
  \item \cmd{ADDR} est une constante definie à la compilation
  \item \cmd{ADDR} pointe sur une entrée d'un tableau initialisé à zero
  \item Chaque entrée du tableau symbolise une ligne
  \item Par conséquent, 100000 ligne augmentra la taille en mémoire de
    l'application d'environ 400Ko
  \item Lorsque le programme se termine, on écrit le tableau sur le
    disque (fichiers \file{.gcda})
  \item Un fichier index est généré durant la compilation afin de faire
    le lien entre les lignes du fichier source et les données générées
    (fichiers \file{.gcno})
  \end{itemize}
  Fonctionne sur une binaire utilisateur. 
  Permet de valider une campagne de test unitaire.
\end{frame}

\begin{frame}[fragile=singleslide]{Couverture de code}
  Utilisation de \cmd{gcov} en natif:
  \begin{itemize}
  \item Compilation
    \begin{lstlisting}
host$ cd x86-cov
host$ gcc --coverage ../hello.c -o hello-cov
    \end{lstlisting} 
  \item Execution d'un scénario d'utilisation
    \begin{lstlisting}
host$ ./hello-cov
    \end{lstlisting} 
  \item Conversion des fichiers  \file{gcda} dans un format lisible
    \begin{lstlisting}
host$ for i in **/*.gcda; do
> gcov -o $(dirname $i) $i
> done
    \end{lstlisting} 
  \item  Convertion  des  fichier  \file{gcda} en  HTML  en  utilisant
    \cmd{lcov}/\cmd{genhtml}
    \begin{lstlisting}
host$ lcov -i -o coverage-x86.info -b .. -d .
host$ genhtml -o html coverage-x86.info
host$ firefox html/index.html
    \end{lstlisting} 
  \end{itemize}
\end{frame}
  
\begin{frame}[fragile=singleslide]{Couverture de code}
  Utilisation de gcov sur la cible:
  \begin{itemize} 
  \item On précise le répertoire ou les fichier doivent être stockés
    \begin{lstlisting}
host$ cd arm-cov
host$ arm-linux-gcc --fprofile-generate=/tmp/hello --coverage
../hello.c -o hello-arm-cov
    \end{lstlisting} 
  \item On execute sur la cible
    \begin{lstlisting}
host$ scp hello-arm-cov root@target:
target$ ./hello-arm-cov
target$ exit
    \end{lstlisting} 
  \item On récupère les résultats pour les traiter sur le host
    \begin{lstlisting}
host$ scp -r root@target:/tmp/hello . 
    \end{lstlisting} 
  \end{itemize}
\end{frame}

\begin{frame}[fragile=singleslide]{Couverture de code}
  \begin{itemize}
  \item Conversion des fichiers  \file{gcda} dans un format lisible
    \begin{lstlisting}
host$ for i in **/*.gcda; do
> arm-linux-gcov -o $(dirname $i) $i
> done
host$ lcov -i -o coverage-arm.info -b .. -d hello --gcov-tool arm-linux-gcov
host$ genhtml -o html coverage-arm.info
host$ firefox html/index.html
    \end{lstlisting} 
  \end{itemize}
  \note{Tester avec -pg et gprof} 
  \note{Tester lcov et genhtml}
  Fonctionne aussi avec le noyau: \texttt{CONFIG\_LCOV}
 
  \note{Parler  de l'alternative  (mais demande  gcc superieur  à 4.0)
    trucov http://code.google.com/p/trucov/}
  \note[item]{TODO: Parler de LTT aussi}
  \note[item]{TODO: Parler de SystemTab}
  \note[item]{TODO: Parler de perf}
\end{frame}

\begin{frame}[fragile=singleslide]{Valgrind}
  On execute notre programme dans une machine virtuelle. Cette machine
  virtuelle
  \begin{itemize} 
  \item instrumente les accès en lecture et en écriture à la mémoire. 
  \item suit les appels aux fonctions \cmd{malloc} et \cmd{free}
  \item suit les copie de pointeurs
  \end{itemize}
  Ainsi, en plus  de détecter les zone non désallouées  à la sortie du
  programme, \cmd{valgrind} est capable de détecter:
  \begin{itemize} 
  \item l'écriture  et la lecture  d'une zone non allouée  (voir aussi
    \emph{DUMA})
  \item la lecture d'une zone jamais écrite
  \item les zone de mémoire n'étant plus pointées
  \end{itemize}
  \cmd{valgrind}  peut  indiquer  les  contextes exacts  où  tous  ces
  évènements se produisent.
\end{frame}

\begin{frame}[fragile=singleslide]{Valgrind}
  \lstinputlisting[language=c,lastline=16]{test-valgrind/hello.c}
\end{frame}

\begin{frame}[fragile=singleslide]{Valgrind}
  Appellons nos deux fonctions de tests
  \lstinputlisting[language=c,firstline=18]{test-valgrind/hello.c}
  Testons:
  \begin{lstlisting}
host$ gcc -Wall -g3 hello.c -o hello
host$ valgrind --leak-check=full ./hello
  \end{lstlisting}
\end{frame}

\begin{frame}[fragile=singleslide]{Valgrind}
  Valgrind a donné naissance à d'autres outils:
  \begin{itemize}
  \item Cachegrind permet de simuler les lignes de cache du processeur
    et  ainsi d'indiquer  ou se  produise les  erreurs de  pages (voir
    aussi \cmd{perf})
  \item Callgrind  permet d'instrumenter  les appels de  fonctions. Il
    génère des graphes d'appels et des informations de profiling (voir
    aussi \cmd{perf})
  \item  Hellgrind  et DRD  instrumente  les  mécanismes générant  des
    sections critique  et détecte  les \emph{Race Conditions}  sur les
    données.
  \item  Massif detecte  les  contexte d'allocation  de mémoire  (heap
    profiling)
  \item Ptrcheck suit les blocks  pointé et détecte ainsi les overruns
    (voir aussi \emph{DUMA})
  \end{itemize} 
  \note[item]{Ne marche pas sous ARM9}
  \note[item]{Montrer une race condition et un overflow sur la pile}
\end{frame}

\section{Profiling}

\begin{frame}[fragile=singleslide]{Bases}
  Avant  de  se  lancer  dans  des procédure  de  profiling  complexe,
  quelques commandes simples:
  \begin{itemize} 
  \item \cmd{free}  indique les ressources mémoire  disponibles sur le
    système
    \begin{itemize}
    \item  Les  \emph{buffers}   indiquent  des  données  devant  être
      écritent  sur le  disque (ou  sur un  IO) mais  dont  le système
      repouse l'écriture pour des raisons de performances
    \item Les  \emph{caches} indiquent des  données lue sur  le disque
      (ou  sur un  IO)  conservées en  mémoires  afin d'accélérer  les
      futurs accès
    \end{itemize}
  \item \cmd{time} indique les ressources utilisées par un processus
    \begin{itemize}
    \item Temps total
    \item Temps CPU en utilisateur et dans le noyau
    \item  Nombre de d'erreur  de page  majeure (nécessitant  un accès
      disque) et mineurs
    \item Nombre de message socket (~réseau) envoyés/reçus
    \item Mémoire utilisée en moyenne
    \item Nombre d'accès aux disques en entrée et en sortie
    \end{itemize} 
  \end{itemize}
\end{frame}

\begin{frame}[fragile=singleslide]{Bases}
  \begin{itemize} 
  \item \cmd{top}  indique à peu  près les même informations  que time
    mais de manière intéractive
  \item \cmd{ps}  affiche aussi ces  informations sur un  ou plusieurs
    processus en cours d'éxecution
  \item    \verb+/proc/${PID}/{stat,statm,status}+   contiennent   les
    données bruts de ces outils
  \end{itemize}
\end{frame}

% \begin{frame}[fragile=singleslide]{gprof}
% \end{frame}

\begin{frame}[fragile=singleslide]{\cmd{perf}}
  \cmd{perf} utilise les registre de debug du cpu. Faible overhead. depend du CPU. Le cpu le plus fourni est x86.
  \note[item]{TODO: Compiler la toolchain avec le support de locale puis compiler libdw et perf}
  \begin{itemize} 
  \item Obtenir une vue synthétique d'un processus
    \begin{lstlisting} 
target% apt-get install linux-tools-common
target$ perf stat ./hello-dbg
    \end{lstlisting}  
  \item Effectuer une mesure plus précise
    \begin{lstlisting} 
target$ perf record ./hello-dbg
target$ perf report
    \end{lstlisting} 
  \item Annoter le code avec les information de profiling
    \begin{lstlisting} 
target$ perf annotate timeattack
    \end{lstlisting} 
  \end{itemize}
\end{frame}

\begin{frame}[fragile=singleslide]{\cmd{perf}}
  \begin{itemize}  
  \item Même manipulation mais avec les contextes d'appel
    \begin{lstlisting}
target$ perf record -g ./hello-dbg
target$ perf report
    \end{lstlisting} 
  \item  Il est  possible d'utiliser  d'autre point  de mesure  que la
    consommation CPU
    \begin{lstlisting}
target$ perf list
target$ perf stat -e cache-misses ./hello-dbg
target$ perf record -e cache-misses ./hello-dbg
target$ perf report
target$ perf annotate timeattack
target$ perf record -e cache-misses -g ./hello-dbg
target$ perf report
    \end{lstlisting} 
  \end{itemize}
  \note[item]{TODO  P4:  Utiliser   des  fichiers  intermédiaire  pour
    différencier la target du host}
\end{frame}

\begin{frame}[fragile=singleslide]{\cmd{oprofile}}
  \begin{itemize} 
  \item Idem \cmd{perf}, mais moins générique
  \item Initialement sur les architectures ST (pas ARM, ni x86)
  \item  Si  vous avez  un  noyau  suffisament récent,  \cmd{oprofile}
    utilise \cmd{perf} comme backend
  \item Démarrage du daemon
    \begin{lstlisting} 
target% opcontrol --init
    \end{lstlisting} 
  \item Liste des évènements pouvant être enregistrés
    \begin{lstlisting} 
target% opcontrol -l
    \end{lstlisting} 
  \item Profiling d'un évènement 
    \begin{lstlisting} 
target% opcontrol --start --no-vmlinux -e CPU_CLK_UNHALTED:100000:0:1:1
target% ./hello
target% opcontrol --shutdown
    \end{lstlisting} 
  \end{itemize}
\end{frame}

\begin{frame}[fragile=singleslide]{\cmd{oprofile}}
  \begin{itemize} 
  \item Génération du rapport
    \begin{lstlisting} 
target% opreport hello
    \end{lstlisting} 
  \item Génération du source annoté
    \begin{lstlisting} 
target% opannotate --source hello-prof
    \end{lstlisting} 
  \end{itemize} 
\end{frame}

% \begin{frame}[fragile=singleslide]{\emph{Tracepoints}}
% \begin{itemize} 
% \item dynamic printk
% \item ftrace
% \item LTT-ng
% \item kprobe
% \item systemtap
% \end{itemize}
% \end{frame}

% \begin{frame}[fragile=singleslide]{slabinfo}
% \end{frame}

% \begin{frame}[fragile=singleslide]{ltt-ng}
% \end{frame}

% \begin{frame}[fragile=singleslide]{Decoder les oops}
% scrips/oops
% \end{frame}


% \begin{frame}[fragile=singleslide]{SAM-BA}
% \begin{lstlisting}
% wget http://www.atmel.com/dyn/resources/prod_documents/sam-ba_2.10.zip
% \end{lstlisting}
% Se connecter par le port USB-Device
% lancer Sam-ba
% \end{frame}

\begin{frame}[fragile=singleslide]{U-boot}
  Notre vendeur nous  fourni une version de u-boot  patchée pour notre
  carte. Dans ce cas, il nous suffit de configurer correctement u-boot
  et de le compiler:
  \begin{lstlisting}
make usb-a9260_config
make CROSS_COMPILE=arm-linux-
  \end{lstlisting}
\end{frame}

\begin{frame}[fragile=singleslide]{U-boot}
  Si nous voulons ajouter une nouvelle carte, nous devons:
  \begin{itemize}
  \item Ajouter un nouveau fichier d'entêtes dans \file{include/configs}
  \item Ajouter un nouveau répertoire dans \file{board}
  \item A priori, vous n'aurez pas besoin d'ajouter une architecture et encore moins un jeu d'instruction
  \item Ajouter une cible dans le Makefile
  \item Cette cible appellera \cmd{mkconfig}
  \item \cmd{mkconfig} prend en  paramètre le nom du fichier d'entête,
    le type  de CPU,  d'architecture et de carte,  ainsi que le  nom du
    vendeur et le modèle de chipset.
    \begin{lstlisting} 
./mkconfig usb-a9260 arm arm926ejs usb-a9260 Calao at91sam926x
    \end{lstlisting} 
  \end{itemize}
\end{frame}

\begin{frame}[fragile=singleslide]{U-boot}
  \begin{itemize}
  \item \cmd{mkconfig} créera  un fichier \file{config.h} pointant sur
    votre fichier entête et un fichier \file{config.mk} contenant:
    \begin{lstlisting} 
ARCH   = arm
CPU    = arm926ejs
BOARD  = usb-a9260
VENDOR = Calao
SOC    = at91sam926x
    \end{lstlisting} 
  \item   Placer   les  drivers   nécéssaire   à   votre  carte   dans
    \file{board/usb-a9260}. L'implémentation des drivers est proche de
    ce que l'on  trouve dans le noyau Linux.  Il suffit d'associer les
    callback aux champs d'une structure.
  \item Le fichier  \file{usb-a9260.h} devra contenir la configuration
    des fonctionnalité de u-boot
  \item Vous trouverez des options de configuration haut niveau telles
    que les commandes devant être incluse
  \item Vous trouverez aussi les paramètre d'initialisation bas niveau
    pour le CPU et l'architecture (timing mémoire, etc...)
  \end{itemize}
\end{frame}

% \begin{frame}[fragile=singleslide]{U-boot}
%   Ca marchera pas. Il faut utiliser celui de Calao. Ou alors, il faut porter notre carte.
%   \begin{lstlisting}
% host$ wget ftp://ftp.denx.de/pub/u-boot/u-boot-2010.09.tar.bz2
% host$ tar xvf u-boot-2010.09.tar.bz2
% host$ make TNY_A9260_config 
% host$ make CROSS_COMPILE=arm-linux-
%   \end{lstlisting}
%   Flasher avec SAMBA pour placer en mémoire avec OpenOCD

%   Dérivés en fonction de architectures : \verb+http://www.denx.de/wiki/U-Boot/Custodians+

%   Ajouter une carte:
%   \begin{itemize}
%   \item Ajouter le fichier .h dans include/config
%   \item Ajouter une cible dans le Makefile
%   \item ``mkconfig ... <votre fichier> cpu architectrure''
%   \item Ouvrir un fichier en exemple
%   \end{itemize}
% \end{frame}


% \begin{frame}[fragile=singleslide]{U-boot}
% Version Calao:
% \begin{lstlisting}
% make usb-a9260_config
% make CROSS_COMPILE=arm-linux-
% \end{lstlisting}
% \end{frame}

% \begin{frame}[fragile=singleslide]{Accès aux paramètres de U-Boot}
%   \begin{itemize}
%   \item Modifier les paramètre de U-boot à partir de Linux
%     \verb+http://www.at91.com/forum/viewtopic.php/f,12/t,4283/start,0/st,0/sk,t/sd,a/hilit,environment/+
%   \end{itemize}
%   \note[item]{TODO: A placer après avoir compilé U-boot}
% \end{frame}

\section{Les sondes Jtag}

\begin{frame}[fragile=singleslide]{OpenOCD}{Qu'est-ce qu'une sonde Jtag?}
  \begin{itemize}
  \item  TAP (Test  Access Port)  est  un module  matériel capable  de
    traiter des  commande de debug (souvent sur  le processeur). Notre
    TAP est embarqué sur le ARM9260EJ-S.
  \item JTAG est un standard de communication normalisé pour les TAP.
  \item  Une ``hardware  interface  dongle'' doit  être utilisée  pour
    empaqueter  le standard  JTAG dans  un autre  protocole (ethernet,
    USB, serie...) afin d'être traitable  par un PC. Nous utilisons la
    FTDI FT2232  pour cette partie.  Cette puce permet de  présenter à
    l'host deux périphérique USB: Le JTAG et la console RS232.
  \end{itemize}
\end{frame}

\begin{frame}[fragile=singleslide]{OpenOCD}{Qu'est-ce qu'une sonde Jtag?}
  \begin{itemize}
  \item OpenOCD permet de gérer  la communication avec la puce FTDI et
    de traduire le langage de bas niveau JTAG en langage de debug gdb
  \item On trouve souvent des sonde JTAG incluant déjà l'équivalent de
    OpenOCD  et sachant  déjà communiquer  avec gdb  (par  ethernet ou
    série très souvent). (Abatron, Lautherbar, etc...)
  \item On peut reconnaitre les deux  types de sondes à leur prix. Les
    première coute au maximum quelques centaines d'euros alros que les
    secondes coutent rarement moins de 1000euros.
  \end{itemize}
\end{frame}

\begin{frame}[fragile=singleslide]{OpenOCD}
  Beaucoup  de sondes  JTAG  (USB, port  parallèle,  port serie)  sont
  supportées                        par                       OpenOCD:
  \file{doc/openocd/JTAG-Hardware-Dongles.html}.

  \note{libftdi et ftd2xx  sont deux bibliothèque permettant d'accéder
    aux  périphériques  FTDI. libftdi  est  open  source, semble  bien
    fonctionner et bien supporté par Linux. ftd2xx est fermé et semble
    mal            supporté           par            Linux           :
    \url{http://forum.sparkfun.com/viewtopic.php?p=78242&sid=1a2a21faa788048ad502179f1598444f}
  }

  Votre distribution inclut peut-être le packet:
  \begin{lstlisting}
host% apt-get install openocd
  \end{lstlisting}
  Il est possible de récupérer le paquet d'une autre disctribution:
  \begin{lstlisting}
host$ wget http://archive.ubuntu.com/ubuntu/pool/universe/o/openocd/openocd_0.4.0-1+nmu1_amd64.deb
host$ wget http://archive.ubuntu.com/ubuntu/pool/main/libf/libftdi/libftdi1_0.18-1_amd64.deb
host% dpkg -i libftdi1_0.18-1_amd64.deb openocd_0.4.0-1+nmu1_amd64.deb
  \end{lstlisting} 
\end{frame}

\begin{frame}[fragile=singleslide]{OpenOCD}
  Evidement, il possible de le compiler
  \begin{lstlisting}
host$ wget http://downloads.sourceforge.net/project/openocd/openocd/0.4.0/openocd-0.4.0.tar.bz2
host$ tar xvf openocd-0.4.0.tar.bz2
host% apt-get install libftdi-dev
host$ mkdir build && cd build
host$ ../configure --enable-usbprog --enable-ft2232_libftdi 
host$ make
host% make install
  \end{lstlisting} 
\end{frame}

\begin{frame}[fragile=singleslide]{OpenOCD}
  Il  est   nécessaire  de  corriger   un  bug  dans  le   fichier  de
  configuration de notre cible:
  \begin{lstlisting}
host% ln -s at91sam9260.cfg /usr/share/openocd/scripts/target/at91sam9260minimal.cfg 
  \end{lstlisting} 
  \note[item]{test avec le fichier mem-ext}
  Nous pouvons maintenant utiliser notre sonde Jtag
  \begin{lstlisting}
host$ openocd -f interface/calao-usb-a9260-c01.cfg
host$ telnet 127.0.0.1 4444
> help
> reset init
> arm reg
> arm disassemble 0x00000000 5
> arm disassemble 0x00000058 5
> ^C
  \end{lstlisting}
\end{frame}

\begin{frame}[fragile=singleslide]{OpenOCD avec gdb}
  \begin{itemize} 
  \item Lancement de \cmd{gdb}
    \begin{lstlisting}
host$ arm-linux-gdb
    \end{lstlisting}  
  \item Connexion avec OpenOCD
    \begin{lstlisting} 
gdb> target remote localhost:3333
    \end{lstlisting} 
  \item \cmd{monitor} permet d'envoyer des commande brutes à OpenOCD
    \begin{lstlisting} 
gdb> monitor reset init
gdb> monitor help
    \end{lstlisting} 
  \item Vu  que nous ne  sommes pas lié  à une binaire,  nos commandes
    sont limitées. Nous pouvons  néanmoins dumper la mémoire et placer
    des breakpoints
    \begin{lstlisting} 
gdb> x/10i 0
gdb> hbreak *0x58
gdb> b *0x0
    \end{lstlisting} 
  \end{itemize}
\end{frame}

\begin{frame}[fragile=singleslide]{OpenOCD}
  Deux mots sur la configuration:
  \begin{itemize} 
  \item   Les   fichiers   de   \file{interfaces/}   contiennent   les
    informations sur le driver à utiliser et comment s'y connecter. Le
    repertoire    \file{target/}   contient   la    configuration   du
    microcontrolleur en lui-même.
  \item        La       documentation       se        trouve       sur
    \url{http://openocd.sourceforge.net/doc/openocd.html}    ou   dans
    \file{/usr/share/doc/openocd}
  \item Il  est possible d'avoir l'aide contextuelle  sur une commande
    \cmd{help command}
    \note[item]{Notre fichier de configurtion  ne déclare pas la flash
      NOR et la flash NAND. Donc, pas possible de flasher}
    \note[item]     {     A     tester:    Activer     dans     u-boot
      :CONFIG\_SKIP\_LOWLEVEL\_INIT et CONFIG\_SKIP\_RELOCATE\_UBOOT}
  \end{itemize}
\end{frame}

\begin{frame}[fragile=singleslide]{Debuguer U-Boot}
  \begin{itemize} 
  \item  Notre fichier  de configuration  de OpenOCD  ne lance  pas la
    cible à la  vitesse maximum.  Laissons cette tâche  à u-boot, puis
    arrêtons la cible.
    \begin{lstlisting} 
Hit any key to stop autoboot:  0 
u-boot> 
    \end{lstlisting} 
  \item Arrêtons la cible afin de pouvoir travailler dessus
    \begin{lstlisting} 
telnet> halt
    \end{lstlisting} 
  \item Configurons une vitesse de transmission raisonnable
    \begin{lstlisting} 
telnet> jtag_hkz 10000
telnet> arm7_9 dcc_downloads enable
    \end{lstlisting} 
  \item Chargeons l'image ELF de u-boot
    \begin{lstlisting} 
telnet> load_image .../u-boot (elf,13secondes)
    \end{lstlisting} 
  \end{itemize}
\end{frame}

\begin{frame}[fragile=singleslide]{Debuguer U-Boot}
  \begin{itemize} 
  \item Vérifions que nous retrouvons bien le code assembleur de U-boot à l'endroit indiqué
    \begin{lstlisting} 
telnet> disassemble XXXX
    \end{lstlisting} 
  \item Sautons jusqu'à cette addresse
    \begin{lstlisting} 
telnet> resume XXX
    \end{lstlisting} 
  \item Arrêtons le démarrage
    \begin{lstlisting} 
telnet> halt
    \end{lstlisting} 
  \end{itemize}
\end{frame}

\begin{frame}[fragile=singleslide]{Debugguer U-Boot}
  Maintenant que notre cible est correctement initialisées, nous pouvons prendre la main dessus avec gdb:
  \begin{lstlisting}
host$ arm-linux-gdb u-boot
gdb> target remote localhost:3333
gdb> c
^C
gdb> hbreak run_command
gdb> c
u-boot> help
\end{lstlisting} 
\end{frame}

% \begin{frame}[fragile=singleslide]{OpenOCD}
%   Beaucoup plus complexe avec le noyau.
%   \begin{itemize} 
% \item \file{vmlinux} souhaite se lancé à partir de l'addresse 0xC0000000. Ceci n'est pas une adresse physique valide. \file{vmlinux} devrait donc être chargé après que le MMU soit démarré
% \item Il est possible de forcer le chargement de \file{vmlinux} autre part en mémoire, mais cela ne permettera pas de booter intégralement
% \item Notre fichier de configuration ne support pas l'usage de DCC avec MMU démarré. En effet, DCC à besoin d'un espace mémoire afin de travailler. Il faut donc une configuration particulière pour le MMU.
% \item 

%   \begin{lstlisting}
% host$ arm-linux-gdb vmlinux
% gdb> target remote localhost:3333
% gdb> monitor halt
% gdb> monitor jtag_hkz 10000
% gdb> monitor arm7_9 dcc_downloads enable
%   \end{lstlisting} 
% \item Si nous ne précision rien, gdb effectura le chargement à l'adresse 0xC0000000 qui n'est pas une addresse physique valide. Nous devons le forcer à charger le noyau à l'addresse 0x21000000
% gdb> load vmlinux -0x9F000000
% \item Le noyau attend que \cmd{r1} lui donne le modèle de la carte
% gdb> set $r1 = 0x44b
% \item 
% ^C
% gdb> hbreak run_command
% gdb> c
% u-boot> help
% \end{lstlisting} 
% \end{frame}
